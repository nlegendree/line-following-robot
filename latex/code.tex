Afin de développer le code, nous nous sommes inspirés des séances de travaux pratiques précédentes. Notamment, en ce qui concerne l'écran LCD et l'I2C, le buzzer et l'utilisation des moteurs. Pour le reste nous avons fait des recherches.

Vous trouverez ci-dessous le programme principal ainsi que les différents programmes annexes :

\subsection{Programme principal}
    \inputminted{c}{code/src/main.c}
\subsection{Défintion de tous les GPIO}
    \inputminted{c}{code/include/gpioPins.h}
\subsection{Gestion du buzzer}
    \inputminted{c}{code/src/buzzer.c}
\subsection{Gestion de la manette}
    \inputminted{c}{code/src/controller.c}
\subsection{Gestion de la distance}
    \inputminted{c}{code/src/distance.c}
\subsection{Gestion de l'écran LCD}
    \inputminted{c}{code/src/i2cLCD.c}
\subsection{Gestion des suiveurs de ligne}
    \inputminted{c}{code/src/lineFinder.c}
\subsection{Gestion des moteurs}
    \inputminted{c}{code/src/motors.c}

\newpage