Afin de développer le code, nous nous sommes inspirés des séances de travaux pratiques précédentes. Notamment, en ce qui concerne l'écran LCD et l'I2C, le buzzer et l'utilisation des moteurs. Pour le reste nous avons fait des recherches.

Vous trouverez ci-dessous le programme principal ainsi que les différents programmes annexes :

\begin{itemize}
    \item Programme principal
        \inputminted{c}{../code/src/main.c}
    \item Défintion de tous les GPIO
        \inputminted{c}{../code/include/gpioPins.h}
    \item Gestion du buzzer
        \inputminted{c}{../code/src/buzzer.c}
    \item Gestion de la manette
        \inputminted{c}{../code/src/controller.c}
    \item Gestion de la distance
        \inputminted{c}{../code/src/distance.c}
    \item Gestion de l'écran LCD
        \inputminted{c}{../code/src/i2cLCD.c}
    \item Gestion des suiveurs de ligne
        \inputminted{c}{../code/src/lineFinder.c}
    \item Gestion des moteurs
        \inputminted{c}{../code/src/motors.c}
    \item Gestion du haut-parleur
        \inputminted{c}{../code/src/speaker.c}
\end{itemize}

\newpage