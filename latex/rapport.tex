\documentclass[12pt]{article}
\usepackage[utf8]{inputenc}
\usepackage{graphicx}
\usepackage{titling}
\usepackage{geometry}

% Marges
\geometry{top=2cm, bottom=2cm, left=2cm, right=2cm}

% En-tête et pied de page
\usepackage{fancyhdr}
\pagestyle{fancy}
\fancyhead{}
\renewcommand{\headrulewidth}{0pt}
\rfoot{\thepage}

\begin{document}

\begin{titlepage}
    \centering
    
    % Logo de l'école
    \includegraphics[width=0.2\textwidth]{logo_insa.png}\par\vspace{1cm}
    
    % Titre
    {\LARGE\bfseries Robot suiveur de ligne\par}
    
    % Sous-titre
    {\Large Projet d'électronique pour systèmes embarqués\par}
    
    \vspace{2cm}
    
    % Noms des étudiants
    \vfill
    \begin{flushleft}
        \textbf{Thomas Bauer}\\
        \textbf{Angelo Boutanou}\\
        \textbf{Ali Hamdani}\\
        \textbf{Nathan Legendre}
    \end{flushleft}
        \vfill
    \begin{flushright}
        % Année universitaire
        \textbf{2023/2024}
    \end{flushright}
    
\end{titlepage}

% Saut de page
\newpage

% Page dédiée à l'introduction
\section*{Introduction}
Dans le cadre du cours Électronique pour les systèmes embarqués nous avons réalisé un projet de robot suiveur de ligne. Le but étant d'appliquer sur un exemple simple et concret les notions de travaux dirigés, travaux pratiques et de cours.

Ce travail a été réalisé en groupe de 4 et supervisé par Mme. Laghmara.

Le projet consiste globalement en un robot qui selon la couleur perçue change de direction ou pas. À cela, nous ajoutons les contraintes suivantes. Le robot doit être capable de détecter un obstacle et de s'arrêter mais également de détecter une intersection. Les informations importantes seront également affichées sur un écran LCD ou alors sonore. Nous avons décidé d'ajouter au robot la possibilité d'être contrôlé via une manette de PlayStation 5.

Afin de mener à bien ce projet, à partir des caractéristiques techniques demandées pour le robot nous avons d'abord réfléchi aux composants adéquats pour réaliser les tâches.

Une fois les composants déterminées nous avons réfléchi à la meilleure façon de relier ces composants au Raspberry Pi (cela sera détaillé dans la section Fritzing) puis nous avons développer les programmes permettant de les faire interagir de la manière souhaitée.

Nous évoquerons évidemment les problèmes rencontrés au cours des semaines de progression sur ce projet et les solutions qui nous ont permis de finaliser le robot suiveur de ligne.

\subsection{Cahier des charges}
\begin{itemize}
    \item Suiveur de ligne
    \item Détection des intersections
    \item Mesurer et afficher la distance frontale avec un objet
    \item Alerter si il y a un obstacle
    \item Arrêt d'urgence si il y a un obstacle. Repartir lorsque la voie est libre
    \item Contrôler manuellement
\end{itemize}

\section{Répartition du travail}
Voici un tableau détaillé de la répartition des tâches pendant le projet.
Le projet aura duré environ 3 mois avec les phases suivantes :
\begin{enumerate}
    \item Sélection des composants en fonction des contraintes imposées
    \item Tests des composants sélectionnés
    \item Développement du code
    \item Montage du robot suiveur de ligne
    \item Tests du robot et résolution des bugs
    \item Rédaction du rapport
\end{enumerate}


\section{Circuit électronique (Fritzing)}

\section{Explication des choix techniques}
Pour la sélection des composants, en plus des éléments imposés par le projet, nous avons décidé de concevoir un modèle 3D, d'opter pour l'utilisation d'une manette SONY PS5 et d'employer des câbles femelle-femelle.

Le choix de la manette s'est imposé en raison de sa précision supérieure par rapport à la télécommande. En effet, les deux gâchettes arrière émettent des valeurs comprises entre 0 et 255, tandis que les joysticks génèrent des valeurs entre 0 et 180. Couplées aux signaux PWM, ces caractéristiques nous ont offert une précision accrue tant au niveau de la vitesse que des manœuvres.

Quant au code de la manette, nous avons initialement utilisé la bibliothèque de bas niveau libevdev, spécifique à la manette PS5. Toutefois, nous avons ultérieurement opté pour la bibliothèque SDL 2.0, plus haut niveau, rendant ainsi notre code compatible avec toutes les manettes.

En ce qui concerne le choix du modèle 3D, il s'est principalement orienté vers des considérations esthétiques. Nous avons privilégié un modèle compact qui évoquerait une véritable voiture télécommandée. Cette orientation a guidé notre décision d'utiliser des câbles femelle-femelle, contribuant à compacter l'ensemble du dispositif.

\newpage

\section{Explication du rôle des composants}
Pour répondre aux contraintes précisées en introduction, nous avons choisi différents composants, parfois plusieurs fois les mêmes. Nous allons expliciter nos choix dans la section ci-dessous :

\subsection{Raspberry Pi}
\subsection{LineFinder}
\subsection{Moteurs et L293D}
\subsection{Capteur d'ultrasons HS-SR04}
\subsection{Buzzer}
\subsection{LCD1602 et I2C Interface Module}
\subsection{Manette de PS5}

\section{Code}

\section{Difficultés rencontrées}
Durant notre projet nous avons rencontré certaines difficultés. La première a été la conception des codes moteurs, le premier composant que nous avions utilisés le MAKERDRIVE était difficile à manipuler et nous n'arrivions pas a envoyé des signaux PWM de valeurs différents. Problème qui n'était pas présent avec le L293D.

La deuxième difficulté a été la réalisation de notre modèle 3D, nous devions pour celui-ci prévoir à l'avance tous les trous pour le fixer et faire passer les fils tout en gardant un résultat esthétique. De plus la limite temps imposé par l'impression 3D nous a poussé à trouver le bon compromis entre un résultat compacte, esthétique et qui nous permette d'y disposer tous les composants.

La troisième difficulté est lié à la deuxième, la limite de temps d'impression et le modèle 3D compact nous a poussé à devoir optimiser l'espace occupé par les composants, en utilisant des câbles femelle-femelle, ressoudant les branches de l'I2C, raccourcir nos cables etc.

Pour finir nous avons trouver une solution à chacun de nos problèmes grâce aux idées de chacun et des conseils des encadrants de TP, ce qui nous a permis d'arriver à un résultat convenable et fonctionnel.

\newpage

\end{document}
