\documentclass[12pt]{article}
\usepackage[utf8]{inputenc}
\usepackage{graphicx}
\usepackage{titling}
\usepackage{geometry}

% Marges
\geometry{top=2cm, bottom=2cm, left=2cm, right=2cm}

% En-tête et pied de page
\usepackage{fancyhdr}
\pagestyle{fancy}
\fancyhead{}
\renewcommand{\headrulewidth}{0pt}
\rfoot{\thepage}

\begin{document}

\begin{titlepage}
    \centering
    
    % Logo de l'école
    \includegraphics[width=0.2\textwidth]{logo_insa.png}\par\vspace{1cm}
    
    % Titre
    {\LARGE\bfseries Robot suiveur de ligne\par}
    
    % Sous-titre
    {\Large Projet d'électronique pour systèmes embarqués\par}
    
    \vspace{2cm}
    
    % Noms des étudiants
    \vfill
    \begin{flushleft}
        \textbf{Thomas Bauer}\\
        \textbf{Angelo Boutanou}\\
        \textbf{Ali Hamdani}\\
        \textbf{Nathan Legendre}
    \end{flushleft}
        \vfill
    \begin{flushright}
        % Année universitaire
        \textbf{2023/2024}
    \end{flushright}
    
\end{titlepage}

% Saut de page
\newpage

% Page dédiée à l'introduction
\section*{Introduction}
Dans le cadre du cours Électronique pour les systèmes embarqués nous avons réalisé un projet de robot suiveur de ligne. Le but étant d'appliquer sur un exemple simple et concret les notions de travaux dirigés, travaux pratiques et de cours.

Ce travail a été réalisé en groupe de 4 et supervisé par Mme. Laghmara.

Le projet consiste globalement en un robot qui selon la couleur perçue change de direction ou pas. À cela, nous ajoutons les contraintes suivantes. Le robot doit être capable de détecter un obstacle et de s'arrêter mais également de détecter une intersection. Les informations importantes seront également affichées sur un écran LCD ou alors sonore. Nous avons décidé d'ajouter au robot la possibilité d'être contrôlé via une manette de PlayStation 5.

Afin de mener à bien ce projet, à partir des caractéristiques techniques demandées pour le robot nous avons d'abord réfléchi aux composants adéquats pour réaliser les tâches.

Une fois les composants déterminées nous avons réfléchi à la meilleure façon de relier ces composants au Raspberry Pi (cela sera détaillé dans la section Fritzing) puis nous avons développer les programmes permettant de les faire interagir de la manière souhaitée.

Nous évoquerons évidemment les problèmes rencontrés au cours des semaines de progression sur ce projet et les solutions qui nous ont permis de finaliser le robot suiveur de ligne.

\subsection{Contraintes du robot suiveur de ligne}
\begin{itemize}
    \item Suiveur de ligne
    \item Détection des intersections
    \item Mesurer et afficher la distance frontale avec un objet
    \item Alerter si il y a un obstacle
    \item Arrêt d'urgence si il y a un obstacle. Repartir lorsque la voie est libre
    \item Contrôler manuellement
\end{itemize}

\section{Répartition du travail}
Voici un tableau détaillé de la répartition des tâches pendant le projet.
Le projet aura duré environ 3 mois avec les phases suivantes :
\begin{enumerate}
    \item Sélection des composants en fonction des contraintes imposées
    \item 
\end{enumerate}


\section{Circuit électronique (Fritzing)}
Après avoir choisi les composants à partir du cahier des charges, nous avons sélectionné les GPIO importants pour envoyer des données à chaque composant.

Afin de schématiser tout cela, nous avons utilisé le logiciel Fritzing qui nous a permis de relier les composants virtuellement assez rapidement.

Vous trouverez le schéma Fritzing ci-dessous.

\bigbreak
\begin{figure}[h]
    \centering
    \includegraphics[width=1\textwidth]{fritzing/schema_composants.pdf}
    \caption{Schéma avec les composants Fritzing}
    \label{fig:Schéma des composants Fritzing}
\end{figure}

\begin{figure}[h]
    \centering
    \includegraphics[width=1\textwidth]{fritzing/schema_electrique.pdf}
    \caption{Schéma électronique Fritzing}
    \label{fig:Schéma électronique Fritzing}
\end{figure}

\clearpage

\section{Explication des choix techniques}
Pour la sélection des composants, en plus des éléments imposés par le projet, nous avons décidé de concevoir un modèle 3D, d'opter pour l'utilisation d'une manette SONY PS5 et d'employer des câbles femelle-femelle.

\subsection{Choix de la manette}
Le choix de la manette s'est imposé en raison de sa précision supérieure par rapport à la télécommande. En effet, les deux gâchettes arrières ainsi que les joysticks émettent des valeurs comprises entre 0 et 255. Couplées aux signaux PWM, ces caractéristiques nous ont offert une précision accrue tant au niveau de la vitesse que des manœuvres.

Quant au code de la manette, nous avons initialement utilisé la bibliothèque de bas niveau evdev, qui utilise directement les fichiers en mode d'accès caractère associés à la manette. Toutefois, nous avons ultérieurement opté pour la bibliothèque SDL 2.0, plus haut niveau, rendant ainsi notre robot compatible avec toutes les manettes.

\subsection{Modèle 3D}
En ce qui concerne le choix du modèle 3D, il s'est principalement orienté vers des considérations esthétiques. Nous avons privilégié un modèle compact qui évoquerait une véritable voiture télécommandée. La base du modèle est une voiture issue du jeu-vidéo Rocket League, qui a pour caractéristique d'être haute et large, rendant le montage des composants internes moins laborieux.

Pour concevoir ce modèle 3D, nous avons modifié un fichier .stl préexistant de la voiture en utilisant le logiciel SolidWorks. Les roues ont été prises comme point de départ pour déterminer les dimensions globales de la voiture, garantissant ainsi une cohérence visuelle avec les garde-boue et les roues fictives de devant. 

En ce qui concerne les composants visibles, tels que le capteur de distance et l'écran LCD, nous avons créé des logements spécifiques dans la carrosserie. Ces logements ont été conçus pour offrir une intégration fluide et discrète, assurant que ces composants s'intègrent naturellement à la conception extérieure de la voiture. Des ajustements minutieux ont été apportés pour garantir la précision et la stabilité du montage, tout en préservant l'esthétique originale du modèle.

Simultanément, nous avons essayer d'enlever le maximum de matière à l'intérieur du modèle afin d'y intégrer les composants internes et de diminuer le temps d'impression. Des supports spécifiques ont été ajoutés pour maintenir en place l'ensemble des éléments électroniques tout en facilitant l'accès pour d'éventuelles réparations ou modifications.
\newpage
Après avoir effectué ces modifications et quelques vérifications sur les dimensions des composants, nous avons envoyé le fichier .stl modifié pour l'impression 3D du modèle final, aboutissant à une voiture radiocommandée compacte, esthétique et fonctionnelle.
\bigbreak
\begin{figure}[h]
    \begin{minipage}[c]{.46\linewidth}
        \centering
        \includegraphics[width=1\textwidth]{solidworks/solidworks1.png}
    	\label{fig:Vue principale conception 3D}
    \end{minipage}
    \hfill%
    \begin{minipage}[c]{.46\linewidth}
        \centering
        \includegraphics[width=1\textwidth]{solidworks/solidworks3.png}
    	\label{fig:Vue de côté conception 3D}
    \end{minipage}
    \begin{minipage}[c]{.46\linewidth}
        \centering
        \includegraphics[width=1\textwidth]{solidworks/solidworks2.png}
    	\label{fig:Vue du dessous 2 conception 3D}
    \end{minipage}
    \hfill%
    \begin{minipage}[c]{.46\linewidth}
        \centering
       	\includegraphics[width=1\textwidth]{solidworks/solidworks4.png}
    	\label{fig:Vue du dessous conception 3D}
    \end{minipage}
    \caption{Captures d'écran de la conception 3D sous SolidWorks}
\end{figure}

\newpage

\section{Explication du rôle des composants}
Pour répondre aux contraintes précisées en introduction, nous avons choisi différents composants, parfois plusieurs fois les mêmes. Nous allons expliciter nos choix dans la section ci-dessous :

\subsection{Raspberry Pi}
\subsection{LineFinder}
\subsection{Moteurs et L293D}
\subsection{Capteur d'ultrasons HS-SR04}
\subsection{Buzzer}
\subsection{LCD1602 et I2C Interface Module}
\subsection{Manette de PS5}

\section{Code}
Afin de développer le code, nous nous sommes inspirés des séances de travaux pratiques précédentes. Notamment, en ce qui concerne l'écran LCD et l'I2C, le buzzer et l'utilisation des moteurs. Pour le reste nous avons fait des recherches.

Vous trouverez ci-dessous le programme principal ainsi que les différents programmes annexes :

\subsection{Programme principal}
    \inputminted{c}{code/src/main.c}
\subsection{Défintion de tous les GPIO}
    \inputminted{c}{code/include/gpioPins.h}
\subsection{Gestion du buzzer}
    \inputminted{c}{code/src/buzzer.c}
\subsection{Gestion de la manette}
    \inputminted{c}{code/src/controller.c}
\subsection{Gestion de la distance}
    \inputminted{c}{code/src/distance.c}
\subsection{Gestion de l'écran LCD}
    \inputminted{c}{code/src/i2cLCD.c}
\subsection{Gestion des suiveurs de ligne}
    \inputminted{c}{code/src/lineFinder.c}
\subsection{Gestion des moteurs}
    \inputminted{c}{code/src/motors.c}

\newpage

\section{Difficultés rencontrées}
Durant notre projet nous avons rencontré certaines difficultés. La première a été la conception des codes moteurs, le premier composant que nous avions utilisés le MAKERDRIVE était difficile à manipuler et nous n'arrivions pas a envoyé des signaux PWM de valeurs différents. Problème qui n'était pas présent avec le L293D.

La deuxième difficulté a été la réalisation de notre modèle 3D, nous devions pour celui-ci prévoir à l'avance tous les trous pour le fixer et faire passer les fils tout en gardant un résultat esthétique. De plus la limite temps imposé par l'impression 3D nous a poussé à trouver le bon compromis entre un résultat compacte, esthétique et qui nous permette d'y disposer tous les composants.

\begin{figure}[h]
    \centering
    \includegraphics[width=0.3\textwidth]{images/solidworks/solidworks2.png}
    \caption{Conception 3D des emplacements vis}
    \label{fig:Conception 3D emplacements vis}
\end{figure}

La troisième difficulté est lié à la deuxième, la limite de temps d'impression et le modèle 3D compact nous a poussé à devoir optimiser l'espace occupé par les composants, en utilisant des câbles femelle-femelle, ressoudant les branches de l'I2C, raccourcir nos cables etc.

Pour finir nous avons trouver une solution à chacun de nos problèmes grâce aux idées de chacun et des conseils des encadrants de TP, ce qui nous a permis d'arriver à un résultat convenable et fonctionnel.

\newpage

\end{document}
