Pour conclure, ce projet de robot suiveur de ligne nous a permis de mettre en pratique nos connaissances acquises en cours d'Electronique notamment sur l'utilisation des composants et du Raspberry Pi. Nous avons également utilisé nos connaissances développées dans le cours d'Algorithme et Programmation en C afin de réaliser le code pour contrôler tous nos composants.

Même si au premier abord, ce projet peut paraître simple, la tâche n'a pas été aussi simple que prévu. En effet, les différentes contraintes ajoutées (contrôler le robot avec une manette de PlayStation 5, créer une coque en 3D) nous ont créer des problèmes que nous avons su résoudre non sans mal.

Évidemment, le résultat n'est jamais parfait mais nous sommes satisfaits du travail rendu et des compétences acquises lors de ce projet.

\subsection{Problèmes à corriger}

\subsection{Améliorations}