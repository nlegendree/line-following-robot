Pour conclure, ce projet de robot suiveur de ligne nous a permis de mettre en pratique nos connaissances acquises en cours d'Électronique notamment sur l'utilisation des composants et du Raspberry Pi. Nous avons également utilisé nos connaissances développées dans le cours d'Algorithme et Programmation en C afin de réaliser le code pour contrôler tous nos composants.

Même si au premier abord, ce projet peut paraître simple, la tâche n'a pas été aussi simple que prévu. En effet, les différentes contraintes ajoutées (contrôler le robot avec une manette, créer une coque en 3D, etc) nous ont créé des problèmes que nous avons su résoudre non sans mal.

Évidemment, le résultat n'est jamais parfait mais nous sommes satisfaits du travail rendu et des compétences acquises lors de ce projet.

\subsection{Améliorations}
Même si nous sommes satisfaits de notre robot suiveur de ligne, il est à noter que nous aurions aimé modifier deux choses sur ce projet mais qu'il était un peu tard pour le faire.

En effet, il aurait pu être intéressant de rendre les moteurs plus puissants en optimisant l'espace à l'intérieur de la coque 3D. Cela nous aurait permis d'utiliser deux petites batteries pour mieux alimenter les moteurs et donc leur permettre de développer plus de puissance en sortie.

Par ailleurs, l'intégration d'un module sonore, tel qu'un haut-parleur, a été envisagée pour enrichir l'expérience pendant le match de foot entre les voitures. Cependant, après une analyse approfondie de la documentation et des essais de codage, nous avons identifié des défis imprévus. La nécessité d'un composant lecteur MP3, occupant un espace supplémentaire, s'est avérée contraignante compte tenu des limitations d'espace dans notre voiture. Malgré nos efforts, la complexité d'intégration a rendu cette fonctionnalité plus difficile à implémenter dans le cadre du projet actuel. Néanmoins, si le temps le permettait, l'ajout d'un haut-parleur aurait offert la possibilité d'intégrer des commentaires sonores et des effets sonores, rehaussant ainsi l'aspect ludique du match.
\newpage