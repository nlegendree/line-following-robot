Pendant notre projet, nous avons fait face à des défis, mais grâce à notre collaboration, nous avons réussi à les surmonter avec succès.

\subsection{Gestion des moteurs}
La première a été la conception des codes moteurs. En effet, le premier composant que nous avons utilisé, le MAKERDRIVE, était difficile à manipuler, pour les commandes de base comme pour tourner à droite et à gauche il fallait envoyer des signaux PMW de valeurs différentes sur notre moteur droit ou gauche en fonction des manoeuvres que nous voulions réaliser, avec le MAKERDRIVE nous n'arrivions pas à envoyer ces signaux.Quant au L293D sur lequel nous avons continué notre projet la présence des pin enable rendait le fonctionnement des moteurs bien plus simple, en nous permettant de fixer une valeur PWM et seulement mettre en marche ou non les moteurs gauches et droites en fonction de la manoeuvre, par conséquent le problème rencontré avec le MAKERDRIVE n'était pas présent avec ce nouveau composant.

\subsection{Fonctionnement du Code}
Durant la phase de développement nous avons rencontré certains problèmes.

L'un des problèmes était une latence entre l'action réalisée sur la manette et la réaction du robot, rendant très difficile sa maniabilité ainsi que certaines manoeuvres. Nous avons alors entâmé des recherches pour trouver une solution à ce problème. Après plusieures tentatives qui n'aboutissaient pas, la solution était plutôt simple. Pour un signal PWM nous avons une clock qui met à jour le signal PWM à chaque période de temps définit par la clock, cette latence ne venait donc ni de la manette ni de la Raspeberry Pi mais bien de la clock PWM, et en la diminuant nous avons réussi à obtenir une maniabilité très bonne avec un réponse presque instantanée du robot.

Un autre problème rencontré pendant notre phase de développement, a été le fonctionnement du mode suivi de ligne. Pour celui-ci nous avions testé le fonctionnement en dehors du circuit final et il fonctionnait correctement, notre table de vérité était correcte mais sur le circuit notre robot effectuait des sortes de zig-zags sur les lignes droites et avait certains problèmes sur les virages. Après avoir revu notre code et notre table de vérité plusieurs fois, nous en avons conclu que le problème ne venait pas d'ici. Nous avons alors observé le comportement du robot sur le circuit et avons remarqué que la vitesse du robot comparée à la taille du circuit impactait le fonctionnement de nos suiveurs de ligne. Nous avons alors réduit la vitesse globale en mode suivi de ligne jusqu'à trouver le parfait compromis entre vitesse et réalisation du circuit.


\subsection{Modélisation 3D}
La troisième difficulté majeure rencontrée lors de la réalisation de notre modèle 3D a été liée aux contraintes spécifiques du fichier .stl que nous avons choisi pour l'impression. Le recours à un fichier maillé comportant plus de 50 000 facettes a rendu l'utilisation de la fonction 'Coque' (qui permet de créer une enveloppe solide autour d'un modèle plein) de SolidWorks impossible. Face à ce défi, nous avons dû entreprendre manuellement le processus de "creusement" en effectuant des enlèvements de matière, une tâche complexe et minutieuse.

En raison du nombre élevé de facettes, le logiciel s'est révélé extrêmement lent, nécessitant plusieurs minutes pour appliquer des fonctions telles que l'enlèvement de matière ou l'extrudage. Cette contrainte a ajouté une complexité significative à la modélisation, exigeant une attention particulière à chaque étape du processus pour garantir la précision et l'intégrité du modèle.

Une autre difficulté notable a émergé lors de la phase d'impression 3D. En raison du temps d'impression trop élevé dû aux supports nécessaires, nous avons été contraints de découper le modèle en trois parties distinctes. Cette démarche visait à optimiser l'utilisation des supports afin de diminuer le temps d'impression, tout en conservant l'esthétique du modèle final.

Par ailleurs, la réalisation du modèle 3D a exigé une anticipation minutieuse de l'emplacement des trous pour la fixation et le passage des fils, nous obligeant à trouver le juste équilibre entre compacité, esthétique et disposition efficace de tous les composants.

\subsection{Taille de notre robot}
La quatrième difficulté est liée à la troisième, la limite de temps d'impression et le modèle 3D compact nous ont poussés à devoir optimiser l'espace occupé par les composants. Ce problème a été un des problèmes majeurs rencontrés durant le projet. Nous avons dû plusieurs fois revoir le montage et innover pour un résultat qui puisse tenir dans la limite d'espace offert par notre modèle 3D.
 
 Parmis les solutions nous avons par exemple décidé de placer la batterie et la pile à l'extérieur du modèle, permettant par la même occasion d'améliorer la maintenabilité du robot. Nous avons également garder uniquement les circuits imprimés de nos composants, revu le choix de nos GPIO pour avoir un câblage plus organisé, utilisé des câbles femelle-femelle,ressoudé les broches de certains composants, raccourcis nos câbles etc.
 
 Ces choix en ont impliqués d'autres. premièrement nous avons rajouter des écrous sur notre roue-libre pour surélever notre robot afin d'éviter le contact de la batterie et la pile avec le sol. Deuxièmement nous avons dû creuser notre modèle 3D après l'impression pour s'adapter à certains composants qui ne tenaient pas. Pour finir même le choix de nos méthodes de fixation ont dû être revues, notamment l'utilisation de scotch double-face à la place des scratchs, vis , pâte à fixe etc car moins volumineux.

\subsection{Isolation des composants}
Après le montage de notre robot, nous avons entamé la phase finale, c'est-à-dire les tests.

Certains tests qui pourtant réussissaient avant le montage ne réussissaient plus. Nous avons alors décidé de démonter le robot afin de trouver la raison de ces disfonctionnements. C'est ici, que nous avons remarqué qu'il y avait certains faux-contacts et courts-circuits dûs à la proximité de nos composants, affectant donc leur bon fonctionnement. Pour y remédier nous avons isolé chaque composant un à un et vérifier si tout était à nouveau fonctionnel.

\bigbreak
Pour finir nous avons trouvé une solution à chacun de nos problèmes grâce aux idées de chacun et des conseils des encadrants de TP, ce qui nous a permis d'arriver à un résultat convenable et fonctionnel.



\newpage
